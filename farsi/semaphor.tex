غالب کتاب های درسی سیستم های عامل در دوره کارشناسی بخشی در همگام سازی دارند که به طور معمول شامل معرفی اجزای اولیه‌ای (موتکس، سمافور، ناظر و متغیر‌های شرطی) و مسائل کلاسیک مثل خواننده نویسنده و تولیدکننده مصرف کننده.
وقتی که من در برکلی کلاسی سیستم عامل را داشتم، و در کالج کالبی این درس را تدریس کردم، به این نتیجه رسیدم که بیشتر دانشجویان قادر به درک راه حل ارائه شده برای اینگونه مسائل هستند، اما تنها برخی از این دانشجویان توانایی [ارئه چنین راه حل‌هایی][ارئه همان راه‌حل‌ها] و  حل مسائل مشابه را دارند.

یکی از دلایلی که دانشجویان به شکل عمیق متوجه نمی‌شوند، به این دلیل است که این مواد زمان زیاد و تمرین بسیاری نسبت به شرایط فعلی موجود، نیاز دارد،
همزمانی یکی از ماژول‌هایی که برای وقت بیشتر می بایست با دیگر ماژول‌ها رقابت کند. و من مطمئن نیستم که بتوانم برای این منظور دلایلی را شرح دهم، منتها من فکر می کنم که سمافور‌ها یکی از چالشی‌ترین و جالب‌ترین و سرگرمی‌ترین بخش‌های سیستم عامل می باشد.

